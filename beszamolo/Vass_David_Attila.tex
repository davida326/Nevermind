\documentclass[a4paper]{article}

% Set margins
\usepackage[hmargin=2.5cm, vmargin=3cm]{geometry}

\frenchspacing

% Language packages
\usepackage[utf8]{inputenc}
\usepackage[T1]{fontenc}
\usepackage[magyar]{babel}

% AMS
\usepackage{amssymb,amsmath}

% Graphic packages
\usepackage{graphicx}

% Colors
\usepackage{color}
\usepackage[usenames,dvipsnames]{xcolor}

% Listings
\usepackage{python}
\usepackage{listings}

% Question
\newenvironment{question}[1]
{\noindent\textcolor{OliveGreen}{$\circ$ \textit{#1}}

\smallskip

\color{Gray}

}{\bigskip}

% Task
\newenvironment{task}[1]
{\noindent\textcolor{RoyalBlue}{$\circ$ \textit{#1}}

\smallskip

\color{Gray}

}{\bigskip}

% Notification
\newenvironment{notification}[1]
{\noindent\textcolor{Peach}{$\circ$ \textit{#1}}

\smallskip

\color{Gray}

}{\bigskip}

% Problem
\newenvironment{problem}[1]
{\noindent\textcolor{OrangeRed}{$\circ$ \textit{#1}}

\smallskip

\color{Gray}

}{\bigskip}

% Solution
\newenvironment{solution}
{\color{RoyalBlue}}{\bigskip}

% Comment
\newenvironment{comment}
{\color{Green}}{\bigskip}

\lstset{% setup listings
        framexleftmargin=16pt,
        framextopmargin=6pt,
        framexbottommargin=6pt, 
        frame=single, rulecolor=\color{black}
}

%===============
\begin{document}
%===============

\definecolor{codebg}{rgb}{0.99,0.99,0.99}
\lstset{backgroundcolor=\color{codebg}}

\begin{center}
    \Large \textbf{Nyári szakmai gyakorlat -- Beszámoló}
    
	\bigskip
	
	\Large Vass Dávid Attila
\end{center}

\medskip

\textit{Helyszín}: Miskolci Egyetem, Alkalmazott Matematikai Intézeti Tanszék

\textit{Konzulens}: Piller Imre

\bigskip

A beszámolóban bemutatásra kerül az elvégzendő feladat, a játék specifikációjának fontosabb elemei, a demonstrációs célból elkészített alkalmazás felépítésének és működésének rövid áttekintése.

\tableofcontents

\section{Az elvégzendő feladat}

A \textit{NeverMind} nevű számítógépes játék specifikálása, Demo-változatának elkészítése Godot játékmotor felhasználásával. A specifikáció a \textit{Sphinx} nevű dokumentáció készítő szoftver segítségével készül. Az elvégzett munkálatok követhetősége érdekében a kapott eredményeknek (dokumentációnak és forráskódnak) \textit{GitHub}-ra, egy külön repozitóriumba kell kerülnie. Az alkalmazás Csehi Mátéval együttműködve készüljün.

\section{Karakterek specifikációja}

\subsection{Attribútumok}

\subsubsection{Élet attribútum}

Az élet attribútum arra szolgál, hogy a felhasználó számára megjelenítse, hogy mennyi sebzést tud még elviselni az irányított karakter, mielőtt meghal.
Játékon belül ez egy \textit{életcsík}ként jelenik meg (\ref{fig:heart}. ábra).

\begin{figure}[h!]
\centering
\includegraphics[scale=0.6]{images/heart.png}
\caption{Az élet attribútum megjelenítési módja a játékban.}
\label{fig:heart}
\end{figure}

Mivel az ellefelek sebzése véletlenszerű ezért a játékosnak folyton figyelemmel kell kísérnie, ha nem szeretne korán elhalálozni.
Az ellenfeleknek is van élet attribútuma, és azoknál is ugyanígy működik.
Amennyiben a játékost sebzés éri, lehetősége van az életét visszatölteni \textit{Health Potion} segítségével.

\subsubsection{Állóképesség/kitartás (Stamina) attribútum}

\begin{figure}[h!]
\centering
\includegraphics[scale=0.6]{images/Stamina.png}
\caption{Az állóképesség attribútum megjelenítési módja a játékban.}
\label{fig:stamina}
\end{figure}

Néhány képesség, mint például a dupla ugrás és a "dash" használata Állóképesség erőforrást használ, így amikor igénybe veszi a játékos ezeket, csökken az Állóképessége.
Ez az attribútum regenerálódik, amint a játékos nem használ ilyen erőforrást igényő képességet.
Itt is, hasonlóan az élet attribútumhoz, a játékon belül \textit{staminacsík}ként jelenik meg (\ref{fig:stamina}. ábra).
Az ellenfeleknek nincs ilyen attribútumuk.

\subsubsection{Mana attribútum}

\begin{figure}[h!]
\centering
\includegraphics[scale=0.6]{images/Mana.png}
\caption{A mana attribútum megjelenítési módja a játékban.}
\label{fig:stamina}
\end{figure}

Ezt az erőforrást a játékos haláltípushoz kapcsolódó képességekkel veszi igénybe.
Minden képesség támadáshoz manát fogyaszt, így figyelembe kell vennie a játékosnak, hogy egyáltalán van-e elég manája, a választott képességhez.
A manaszint regenrálódik abban az esetben, ha a játékos aktuálisan nem használ manát igénybevevő képességet.
Ez is (mint az Élet és Állóképesség attribútumok esetében) növelhető ugrade-ek vásárlásával.
Az ellenfeleknek nincs ilyen attribútumuk.

\subsubsection{Gold}

Az aranynak a játékban fontos szerepe van mivel, ezt felhasználva vásárolhat a játékos az ördög NPC-től upgrade-eket.
Lehetősége van a játékosnak képesség és attribútum növelő fejlesztések vásárlására.
Aranyat a játékos a legyőzött ellenségektől, küldetések teljesítésével és a pálya egyes, rejtett részein szerezhet.

\subsubsection{Kulcsok}

A játékosnak az előrehaladás érdekében kulcsokat kell gyűjteni speciális ajtók és ládák kinyitásához.
Kulcsokból négy féle változat létezik: piros, kék, zöld, sárga.
Néhány ajtóhoz több különböző színű kulcs szükséges, hogy az ajtót ki lehessen nyitni.
A kulcsok a pályán meghatározott helyeken el vannak rejtve, amiket a játékosnak meg kell találnia.
A játékos egy színből egy kulcs lehet, így összesen négy különböző színű kulcsot tarthat magánál, amiket később felhasználhat ajtók kinyitásához.

\subsection{Típusok}

Egy fő karakter típus van. Haláltípustól függően a karakternek megváltozik a kinézete és a képessége.
Ellenfélből két féle lesz: egy közelharci és egy távoli sebzési móddal rendelkező.

\subsubsection{A főhős}

A játék során a főhős az egyetlen irányítható karakter (\ref{fig:programmer}. ábra). Közelharci támadása bármilyen haláltípusnál használható, ami nem igényel manát.
A közelharci támadáshoz legalább 1 block közel kell állnia a játékosnak az ellenféltől.

Közelharci támadást minden haláltípus karakterrel végre lehet hajtani.
A programozó karakter speciális képessége a bögréjének elhajítása.
Ez elsőre nem tűnhet félelmetes támadásnak, ám igen hatékony, mivel a programozó karakternek nagy gyakorlata van benne.

\begin{figure}[h!]
\centering
\includegraphics[scale=3]{images/programmer.png}
\caption{A játék főhőse, a Programozó, bögrével a kezében.}
\label{fig:programmer}
\end{figure}

Az attribútumai a következők.
\begin{itemize}
\item Sebzés: Bögre hajítás (1-3)
\item Élet: 12
\item Állóképesség: 12
\item Mana: 12
\item Mozgási sebesség: 300(Max)
\item Karakter inventory: arany (helyi fizetőeszköz), kulcsok (maximum 4 db különböző színű kulcs lehet nála)
\end{itemize}

A különböző halálképességeinek sebzése 1.5-3.5 (kezdetben, ez később fejleszthető), ezeknek használata 3 manapont.
A képességeket 2 másodperc lehülési idő elteltével használhatja.

\subsubsection{A gépfegyveres kolléga}

Ellenséges NPC, általában őrt áll valahol vagy éppen járőrözik.
Amint a játékos bekerül a látóterébe, a kolléga megtámadja őt.
A játékosnak vigyáznia kell vele mert ő távolról is képes igen nagy sebzést okozni. Érdemes fedezékből fedezékbe közelíteni hozzá.
Az kollégának 15 block távolságú a látótere.

Attribútumai:
\begin{itemize}
\item Sebzés: 2-4
\item Élet: 8
\item Mana: nincs
\item Állóképesség: nincs
\end{itemize}

Amikor támad a játékosra, és az túl közel kerül, megpróbál elmenekülni és távolról újra felvenni vele a harcot.
(megpróbál 10 block távolságra lenni tőle és újra tüzel)
Mozgási sebesség: 250 (Max)

\subsubsection{Lángszórós rosszfiú}

Ellenséges NPC. Amint észreveszi a főhőst, megpróbál közel kerülni hozzá és a lángszórójával elégetni.

Attribútumai:
\begin{itemize}
\item Sebzés: 1-3
\item Élet: 10
\item Mana: nincs
\item Állóképesség: nincs
\end{itemize}

A játékos felhasználót 15 block távolságról kiszúrja, és ha nincs takarásban, elindul felé hogy támadjon.
Mozgási sebesség: 150 (Max)

\subsubsection{Ördög}

%devilinhell.png

Semleges NPC, a főhős főnöke.
A játékos tőle tud vásárolni fejlesztéseket a speciális képességeihez, illetve passzív upgrade-eket is vehet, mint például Élet, Mana, Állóképesség növelés.
A főhős, az "E" interakció gombbal tud vele kommunkálni.
Gyakori monológja hogy: "Everything has a price..."

\section{A főhős halálképességei}

\subsection{Repülő halálképesség}

A halálképességek beáltakor, ugyanazzal az élet/Állóképesség/mana attribútum mennyiséggel osztoznak a különböző karakterek.

A főhőst, ha magas zuhanás általi halál éri, ez a haláltípus áll be nála.
Passzív képességének tekinthető, hogy tud repülni és emiatt többé képtelen meghalni árokba zuhanástól.

Speciális képességéhez a karmait használja, amit mélyen az ellenfélbe mélyeszt.
Ennek a képességnek az ára 1 manatöltés.
Miután használta a képességet, jelentkezni fog egy lehülési idő, ami 2 másodperc, ezt követően tudja újra használni a képességet, abban az esetben ha van elegendő manája.

\begin{itemize}
\item A képesség sebzése: 4
\item Mozgási sebesség: 450 (Max)
\end{itemize}

\subsection{Tűz halálképesség}

Abban az esetben veheti fel a játékos ezt a képességet, hogyha tűz általi halált hal.
Ezt robbanó hordók, Lángszórós rosszfiú karakter tudja előidézni.
Ekkor a játékos egy láng démonná alakul, és emiatt többet nem sebezheti a tűz/robbanás.

Speciális képessége, hogy egy tűzgolyót tud lőni az ellensége felé.
A képesség használata után 2 másodperc lehülési idő lép életbe.

\begin{itemize}
\item A képesség sebzése: 3-4
\item Mozgási sebesség: 300 (Max)
\end{itemize}

\subsection{Vas halálképesség}

Ezzel a képességgel akkor rendelkezik a játékos, hogy ha a gépfegyveres kolléga túl sokszor eltalálja és amiatt meghal.
Főhősünk ezzel a halálképességgel fog újraéledni.

Speciális képessége, hogy hosszú karjait használva egy erőütést tud végrehajtani.
A képesség használata után 2 másodperc lehülési idő, lép életbe.

\begin{itemize}
\item A képesség sebzése: 3-5
\item Mozgási sebesség: 200 (Max)
\end{itemize}

\section{Az alkalmazás elkészítése}

A játék világának az elemei saját készítésűek/szerkesztésűek. A Godot nevű játékmotor és annak a GodotScript nevű programozási nyelve került felhasználásra. A grafikus elemek tervezése és magának az alkalmazáslogikának a megírása is közvetlenül a Godot fejlesztőkörnyezetében készült.

A játék elemei az OOP elveknek megfelelő külön modulokba kerültek. Például az ajtó definíciója a \texttt{door.gd} állományban került részletezésre, amely tartalma a következőképpen néz ki.
\begin{python}
extends Node2D

var area = false
var player = null

func _process(delta):
    if area:
        if Input.is_action_just_pressed("ui_interact"):
            $Area2D/AnimatedSprite.play("open")
            player.global_position = get_parent().get_node(self.get_name()).
                get_node("Position2D").global_position
            player.global_position += Vector2(16, 34)

func _on_Area2D_body_entered(body):
    if body.get_name() == "programmer":
        player = body
        if body.get_node("Camera2D").get_node("key_bar").
            get_node($Label.text).visible:
            area = true

func _on_Area2D_body_exited(body):
    if body.get_name() == "programmer":
        player = null
        area = false
\end{python}

A Godot egy eseményvezérelt környezetet biztosít.

A program fontosabb szerkezeti elemei így a szkriptek (\texttt{scripts}), színterek (\texttt{scenes}), textúrák (\texttt{textures}) és tile szettek (\texttt{tile\_sets}).

\end{document}
